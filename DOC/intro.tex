\section{Introduction}
The available of new firm-level detailed data has promoted a great amount of research which study innovation, exports and firm performance. The main findings are consistent with the idea of positive relationship between: exports, productivity and innovation [\cite{BernardJensen1999}; \cite{CDM} and \cite{LopezRodriguez2005}]. In relation to innovation activities, while some firms engaged in R\&D projects and new technology creation (mainly firms from developed countries), most of them simply imitate or adapt existing production techniques to local conditions [\cite{EvensonWestphal1995} ; (UNCTAD, 1999)]. In developing countries, the former idea is important because the main source of technological progress is related to technological adoption from developed countries, rather that in-house R\&D [ (Hoekman, Maskus, \& Saggi, 2004); (UNCTAD, 2004)]. Furthermore, study innovation doing the distinction between technological adoption and R\&D is important, especially for developing countries. 
This paper aims to provide evidence, for developing countries, of the causal relationship between the innovation effort and the export propensity, considering how this causal relationship change under technological adoption or R\&D expenditures. Which is a contribution regarding previous papers in two main aspects. 
First, there are few studies that take into account the firm-level export-innovation relationship in developing countries, and those few research that do it, take into account only R\&D expenditure as innovation and they do not consider acquisition of knowledge [ (Benavente, Ortega-Bravo, \& González, 2013); (Álvarez, García, \& García, 2008); (Şeker, 2012); (Álvarez \& Robertson, 2004) and (Almeida \& Fernandes, 2007)]. In that sense, my paper study an interesting aspects for the mainly low-tech firms from developing countries. Firms in developing countries innovate, but not in a sophisticated-disruptive way like developed ones. They do not have large R\&D departments and laboratories; rather, they adapt external knowledge and technology, in the form of patents, license, and new machinery for innovation. They performs little improvements to their products in order to achieve new market requirements and expand, the change the shape of products and not their technical specifications. Furthermore, firms’ innovations are mainly related to other expenses, like acquisition of knowledge, rather than R\&D.   


Secondly, there is not literature that study innovation-export causal relationship considering both R\&D and acquisition of knowledge separately. For one side, there is a considerable amount of literature—mainly in developed countries—that study the effects of innovation on firms’ exporting behavior (Barrios, Görg, and Strobl 2003; Cho and Pucik 2005; Díaz-Díaz, Aguiar, and Saá-Pérez 2008; Kyläheiko et al. 2011; Vila and Kuster 2007; Basile 2001; Cassiman and Golovko 2010 and Wakelin 1998). In the other side, some literature has examined the reverse relationship—namely, the effect of exports on firms’ technological resources and innovation (Golovko and Valentini 2011; Hitt, Hoskisson, and Kim 1997). Internationalized firms are able to maintain their international competitiveness by acquiring more experience and technological knowledge in foreign markets (Zahra, Ireland, and Hitt 2000). These papers examine only a single causal direction of innovation-export relationship (Cho and Pucik 2005; Damijan, Kostevc, and Polanec 2010; Kyläheiko et al. 2011), and have not consider the double relationship (Kumar and Saqib 1996; Salomon and Shaver 2005; Zahra, Ireland, and Hitt 2000). With only a few notable exceptions (Filatotchev and Piesse 2009; Golovko and Valentini 2011; Monreal-Pérez, Aragón-Sánchez, and Sánchez-Marín 2012), which jointly examine innovations and exports without define the causality relation previously. Golovko and Valentini’s (2011, p. 375) study highlights a “dynamic virtuous circle” with regard to innovation and exports, arguing that they are “complementary activities that reinforce each other, and whose individual marginal contribution to [small and medium-sized enterprises’] sales growth is higher if the other activity is also in place.” Their study complements that of Fila- totchev and Piesse (2009) by examining the joint effect of innovation and exports over small and medium-sized enterprises’ growth. 
Filipescu (2015) study the double causal effect between a firm’s export and innovation activities, which has been overlooked insofar as they have typically been related to one another unidirectionally (Pla-Barber and Alegre 2007; Vila and Kuster 2007).
None of these study the relationship between the innovation effort and the exports propensity, considering how this causal relationship change under technological adoption or R\&D expenditures.
