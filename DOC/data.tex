\section{Data and Descriptive Statistics}
The data source that was used to study innovative activities in Chile at the firm level is, first and foremost, the Technological Innovation Survey (EIT, for its acronym in Spanish) conducted by the National Statistics Institute (INE, for its acronym in Spanish). This survey includes a questionnaire that follows the guidelines set forth by the Frascati Manual regarding innovation, which was published by the OECD. Our study takes into account the eight innovation surveys that were conducted from 1995 until 2012. We only studied manufacturing companies in order to have more homogeneous observations. 

The studied data contains more than 6,300 observations.  As was previously mentioned, we considered a wider range of activities that seek productive improvements inside the firm. Table 1 shows the different types of innovation expenditure considered by the data that we used.

TABLE 1


Regarding the remaining variables, Table 2 show descriptive statistics, where we can observe the percentage of companies that carry out some sort of innovation expenditure (generally speaking), companies that export, companies that carry out both activities simultaneously, and lastly companies whose property is partly owned by foreigners.
We can notice that the amount of firms that innovate  is close to 40\%, which is similar to the amount of firms that export. However, if we take into account both firms that innovate and export at the same time, we see that they make up 22\% of the sample, so we can reach the conclusion that approximately half of exporting firms innovate. The same situation happens with innovative companies; that is, half of them export. Finally, 11\% of firms have foreign property. 

TABLE 2

In order to study the innovation-export relationship more thoroughly, Table 3 shows a Mean Test of innovation expenditure effort for exporting and non-exporting companies. To carry out this test, innovation effort is defined as the innovation expenditure per worker.  As you can see, there are statistically significant differences between both groups of firms if the whole sample is considered at the aggregate level, where exporting companies have a higher innovation effort level than non-exporting companies. This evidence is coherent with previous studies, and we are able to confirm that exporting companies have a (statistically significant) better innovation performance than non-exporting companies. We observe a difference (difference between means) of more than 750 and a proportion between means  (ratio) that is more than double. 

TABLE 3

In order to study the export intensity relationship between innovative and non-innovative firms, we carried out a Mean Test regarding exporting propensity, measured as the proportion of sales intended for foreign markets. The results are shown in Table 4. We can observe statistically significant differences between both groups of firms if the whole sample is considered, where innovative companies, on average, have a higher export intensity level than non-innovative firms, with a 7\% difference (dif) and a proportion between means  (ratio) of almost 2 (1.8). That is, the percentage of exported sales of innovative firms is almost twice that of non-innovative firms, which is also significant at the sub-sector level, reaching ratios between 1.3 and 3. This relationship is coherent with previous studies, and thus we can prove that exporting companies have a better innovation performance than non-exporting companies. 

TABLE4






