\section{The Model}
I analyze the causality of 2 relevant variables: innovation expenditure effort and export intensity. We control this estimate for productivity to capture economies of scale, scope and all possible existing relationship of learning-by-exporting and censured variables. This is achieved by distinguishing between the types of innovation expenditure. In the first place, we study general innovation expenditure (g); that is, all the expenses that are presented in Table 2. Secondly, we consider technological innovation expenditure (g1), defined as the sum of all Expenditure on machinery for innovation and on other Activities (installation and adjustment of new equipment, commissioning of production) for Innovation. In the third place, we consider expenditure on acquisition of external knowledge (g2). Finally, we consider only Research and Development Expenditure (g3). The objective of differentiating between kinds of expenditure is to look for heterogeneities in the causal behavior related to the level of sophistication of the type of innovation expenditure.

We mainly followed the methodology used in the most recent paper on this topic, which studies the causality between innovation and exports  \citep{Filipescu2013}. This paper studies causality by means of a Granger Test \citep{Granger} with two lags. Along with the previous paper, we consider the analytical approached used by \cite{MonrealPerez2012}, which also controls for productivity in order to study the causal relationship between innovation and exports. In order to study the sub-sector heterogeneity, we make estimates for every sub-sector separately.

The study was carried out based on pooled cross-sectional data, estimating a Granger test based on one lag. Considering the manufacturing sub-sectors (s), the model specification was established in the following manner for every sub-sector:

EC1
EC2

These innovation and export equations are estimated by means of a Tobit model \citep{Amemiya}, considering that both exporting firms and firms that carry out innovation efforts are censured samples.

The variables are calculated in the following manner: the different amounts of innovation expenditure are calculated according to the aforementioned definition, and afterwards the amount of innovation expenditure per worker is calculated, which is called innovation effort. Exports are considered to be the real value of total exports. This value is divided by the total sales, and the result is the company's export intensity. Finally, labor productivity is measured by means of the number of sales per worker. 

In order to correct the effect of the year in particular in which the variable was measured, we used categorical variables (A) to capture the variance stemming from the year of observation. This is relevant to lessen the different temporal exogenous effects that there may be as much as possible. Clear examples are the Asian financial crisis and the sub-prime mortgage crisis, which happened in 1998 and 2009, respectively, as well as the fact that, as we saw earlier, the amount of exporting and innovative companies might by permeable to macroeconomic tendencies (Real Exchange Rate). Table 5 displays a summary of the variables that were used along with their definition.

Table 5