\section{Literature Review}
As I mentioned in the previous section, productivity, exports and innovation are highly correlated. Therefore, in order to isolate the export-innovation causality, we must thoroughly correct the inherent effect on productivity in such a way that the export-innovation causality may be identified in isolation, which is the focus of this study. For this reason, I will begin by reviewing literature that studies the productivity-export and productivity-innovation relationships, in order to subsequently study the export-innovation relationship.

\subsection{Export causality towards innovation and productivity}
There is plenty of empirical evidence that shows that exporting firms have a higher productivity than firms that only participate in domestic markets. Among these studies, Jensen and Bernard's paper (Bernard \& Jensen, 1999) stands out as the most renown on this topic. This paper —- and studies that have stemmed from it -- shows that exporting companies are bigger, more productive, more capital-intensive, have greater human capital, pay higher wages and invest more on technology and R\&D \citep{BenaventeBravoGonzalez2014}.

In terms of the causality between exports and productivity, there are hypotheses that aim to explain this phenomenon in different ways. A first approach is referred to as learning-by-exporting, which says that companies increase their productive performance only by exporting. \cite{GreenawayKneller2007} argue that this approach considers at least three causality mechanisms:

\begin{itemize}
\item[1.] On the one hand, the knowledge and technology that firms absorb in international markets, which non-exporting companies do not have access to, will increase exporting firms' performance and, consequently, their productivity.
\item[2.] Exporting firms have access to a global market that is bigger than the local market, and so their production will benefit from economies of scale.
\item[3.] Firms participating in global markets are subject to a greater competition level, which will force them to invest in innovation in order to reach greater efficiency levels.
\end{itemize}

\cite{DeLoecker2007} finds evidence of a positive effect on new entrants' productivity after they have begun their exporting activities. Through a data sample from Slovenia, he also concludes that the destination of exports is an important factor in determining learning-by-exporting.

\subsection{Innovation and export causality towards productivity}
It is argued that firms that have access to global markets must have previously improved their performance, particularly their productivity. It is to be expected that firms with better performance in terms of their productivity may access and continue participating in global markets, where worldwide firms participate and which, in turn, previously improved their performance. This way, a greater competition level will exist in global markets. This hypothesis, which states that only firms with certain characteristics will access export activities, is called self-selection and may be linked to certain entrance expenses (such as those regarding transportation, distribution, skilled personnel to handle the international network, information asymmetries regarding quality, etc.), expenses that would only allow firms that have innovated and increased their productivity to have access to the competitive export markets.

\cite{BernardJensen1999} is the most renowned work studying this self-selection. Using USA data, the authors find evidence that firms that become exporters are successful before beginning their export activities. Wagner (2008) studies compiles evidence in the same manner.

\cite{MonrealPerez2012} studies the causal relationship between innovation and exports, and whether innovation improves firms' ability to export more products or if exporting propensity is what leads companies to innovate. Additionally, the authors study how productivity changes this relationship. Using a database with 14,142 observations from Spanish manufacturing firms during the 2001-2008 period, we can see that innovation prompts firms to increase their exporting propensity, a result that proves to be sturdy in the face of endogeneity tests. However, the test does not conclusively demonstrate that exports generate learning through product or process innovation. Lastly, it does not show that productivity modifies any of these relationships.

\subsection{Mutual reinforcement between exports and innovation}
The relevance of innovation at the firm level has been analysed in several studies, which have found that the development of abilities to innovate, which are endogenous to firms, is one of the main incentives that allow firms to export \citep{Leonidas2007}. Most of these studies have suggested that innovation positively affects the development of exports by showing that innovation can set a firm apart and therefore constitute a source of competitive advantages in international markets \citep{LopezRodriguez2005}, due to the fact that innovation acts as a source of competitive advantages that are hard to imitate. A firm's ability to innovate represents a combination of the organization's resources that has been developed throughout the firm's existence. To imitate innovations is difficult, given that competitors may not possess the necessary resources to exploit these abilities \citep{MillerShamsie1996}. In turn, firms that have developed a certain innovation will have incentives to exploit said innovation in larger markets and in different markets in the search to improve their economic performance \citep{PlaBarberAlegre2007}.  

Furthermore, it has been argued that the high competition level in global markets forces firms to constantly update their products and adapt to these markets' new conditions. \cite{SalomonShaver2005} study Spanish firms and found that the knowledge that is gained in international markets allows firms to register an even higher amount of patents and to develop more innovative products in general. The authors emphasize the importance of studying how long it takes for Learning-by-Exporting to have an effect, given that this effect may not be immediate. The authors find that Learning-by-Exporting first affects product innovation two years after the firm has begun exporting and that the amount of patent filings increases with a much greater lag. On the other hand, \cite{Silva2007}, in a study conducted about Portuguese manufacturing firms, come to the conclusion that firms with higher export levels are less able to innovate. The authors argue that most companies with high export intensity outsource services and adopt a low-price strategy, which does not correspond with product innovation. 

In a more recent study, \cite{Filipescu2013} investigate how innovation (R\&D intensity, product and process innovation) on the one hand and exports (breadth and depth) on the other hand can mutually influence each other. The causality between both effects is examined by means of a panel of 696 Spanish manufacturing firms during the 1994-2005 period. There is evidence of a reciprocal relationship between technological innovation and exports (mutual causality). In turn, the authors find positive, but not significant, connections between product innovation and exports, as well as between export depth and process innovation. The authors argue that these results are consistent with most existing international studies on this topic. This relationship of reciprocal causality could be centred on the relevance of resources and learning based on the development and use of intangible resources. As companies develop export activities, they gain knowledge and abilities that help them develop new technological innovations, which in turn allow exporting firms to increase their export intensity on the one hand and diversify the global markets in which they participate on the other hand.

\subsection{Evidence for Chile}
In the case of Chile, \citep{AlvarezAgo2008} study the relationship between exports, productivity and technological productivity. By means of information from manufacturing plants, they explore what factors might explain the positive relationship between exporting performance and their productivity. The approach that the authors use to identify if Chilean companies that begin to export become more productive was implemented by means of matching techniques\footnote{Exporting companies may be slightly biased to be more productive than the average company, given that precisely more productive companies would self-select themselves to export. This problem is known as selection bias and, in order to control for these and other non-observable variables, endogeneity treatment and selection bias methodologies must be used in order to capture the desired effects efficiently and without biases.}. The evidence that the authors found favors the idea that only more productive firms are able to export. This Self-Selection phenomenon could be explained through the existence of prices for trading with the rest of the world. Given the small amount of evidence that points towards the fact that exports increase firms' productivity, this paper analyzes whether there are other kinds of learning linked to the exporting process: using information regarding the location of plants and their productive sector, there is evidence regarding whether the probability of entering international markets is affected by the existence of other exporters in the same region and/or sector. The results for Chile do not suggest a significant profit stemming from the high geographic and sectorial concentration of exporters. Finally, the authors analyze whether firms' innovation activities favor their exporting performance, and find no evidence that these activities increase their probability of exporting.

\cite{BenaventeBravoGonzalez2014} study the relationship between R\&D expenditure, exports and productivity. The authors find that companies that invest in R\&D are considerably more prone to export, but not the opposite situation; that is, the fact that a firm exports does not increase the probability that it carries out research and development. Furthermore, there is evidence that exports and R\&D have a joint effect on increasing Chilean manufacturing plants' productivity levels. In this manner, although exporting activities do not encourage R\&D investment, there is evidence of learning-by-exporting. All these results are sturdy in the face of endogeneity corrections.

Most of the studies mentioned have used dummies to capture sub-sector differences. However, this methodology does not reveal sub-sector distinctiveness. The main focus of this paper is to analyze how innovation affects exports, and vice versa. In order to achieve this, we have estimated the relationships of causality for every sub-sector separately,  which allows us to evaluate to what extent existing differences between the various sub-sectors condition the relationship of causality between exports and innovation.

Our assumption in this paper is that these differences (which include the following: differences regarding technological intensity, aggregate value level, whether the sub-sector represents a natural-resource intensive sector or not, the internalization of companies that belong to the sub-sector, and the location where the firms from one or another sub-sector are found, among other differences) have a heterogeneous sub-sector impact on the relationship of causality between exports and innovation, an assumption that seems relevant when considering Pavitt's important contributions regarding sector specificity on innovation sources and effects \citep{Pavitt1984}.
